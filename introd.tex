\chapter*{Introducció}
\setlength{\epigraphwidth}{0.7\textwidth}
\epigraph{Fui consciente de la presencia de algo con lo que tenía un claro parentesco, incluso en situaciones que solemos considerar salvajes y temibles, y también de que lo más próximo a mí en sangre y más humano no era una persona ni un ciudadano, de modo que pensé que ningún lugar podía resultarme extraño en adelante.}{\textnormal{Henry David} \textsc{Thoreau}, \textit{Walden}}

% Abans de començar, una petita introducció a aquests apunts pels quals, a causa de la seva extensió, es pot veure fàcilment que he ampliat més que no pas reduït el temari... 
Primer de tot, es trobarà que hi ha un índex, on hi distingim els diferents apartats ordenats d'una manera certament poc satisfactòria pel que fa a l'ordre cronològic del curs, sinó que he seguit més aviat el meu propi criteri. Hi ha capítols, seccions, subseccions (i fins i tot subsubseccions). Pel que fa al pes d'aquestes estructures en els encapçalaments:
\begin{enumerate}
    \item el número de l'última secció/subsecció figurarà en cada cantonada superior de pàgina parella (per exemple, \textit{1.2});
    \item el nom del capítol es trobarà a la part dreta de la capçalera de les pàgines parelles (per exemple, "Divisibilitat i nombres primers");
    \item el nom de l'última secció/subsecció de la pàgina, a la cantonada dreta superior de les pàgines parelles (per exemple, "Polinomis: algorisme d'Euclides").
\end{enumerate}
A més, hi ha una taula en què es veu fàcilment que s'ha seguit una mena de \textit{sorting-by-color} per poder treballar de manera més eficient amb els diferents tipus d'enunciats matemàtics. En els encapçalaments, aleshores, tenim:
\begin{itemize}
    \item el número de l'últim teorema, definició... de la pàgina en qüestió es trobarà a les pàgines senars, a la cantonada superior dreta (per exemple, \textcolor{arsenic}{\textbf{1.2.3}}).
\end{itemize}
